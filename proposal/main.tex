\documentclass[letterpaper,twocolumn,10pt]{article}

\usepackage{usenix-2020-09}

\usepackage{fontspec}

\usepackage{amsmath}
\usepackage{amssymb}
\usepackage{mathtools}

\usepackage{xurl}
\urlstyle{same}

\usepackage[capitalise]{cleveref}
\crefformat{section}{\S#2#1#3}
\crefformat{subsection}{\S#2#1#3}
\crefformat{subsubsection}{\S#2#1#3}
\crefname{figure}{Figure}{Figures}

\setmonofont{Fira Code}[
  Contextuals=Alternate  % Activate the calt feature
]

\usepackage{lstfiracode}
\usepackage{listings-rust}
\lstset{
  language=Rust,
  style=FiraCodeStyle,
  basicstyle=\ttfamily\footnotesize,
  numbers=left,
  tabsize=2,
}

\newcommand{\mlstinline}[1]{\ensuremath{\text{\lstinline{#1}}}}

\newcommand{\Hermit}{\textsc{Hermit}}

\begin{document}
\ActivateVerbatimLigatures

\date{}

\title{\Large \bf Hermit: Static Information Flow Control for Rust via Dynamic Epistemic Logic}

\author{
  {\rm Eric Zhao} \\
  Brown University
}

\maketitle

\section{Introduction}

Epistemic (from the Greek \emph{epistēmē}, ``knowledge'') logics present formal systems for reasoning about knowledge,
belief, and related notions. Whilst of intense philosophical interest, these logics have also been
the subject of significant practical interest in computer science and other fields, especially in
artificial intelligence and multi-agent systems. We suggest that they also provide an intuitive
framework for reasoning about notions of privacy and security: policies in this realm are often
concerned with \emph{who} may or must \emph{know} some information.

This is, unsurprisingly, not a novel observation, and there has been plenty of investigation in
security applications of epistemic logic \cite{aucher_2011a, dechesne_2010a, balliu_2011a,
koleini_2013a, soloviev_2024a}. In this work, we employ \emph{dynamic epistemic logic} in
language-level information flow control. The result is \Hermit{}, a library for \emph{static}
information flow control in the Rust programming language.

\section{Dynamic Epistemic Logic}
\label{sec:background-del}

Epistemic logics are most commonly modal logics, extending propositional logic with modal operators
for \emph{knowledge} (\(K\)) and \emph{belief} (\(B\)), corresponding to modalities \(\square\)
(``necessity'') and \(\diamondsuit\) (``possibility''). Here, we are only concerned with the former,
and knowledge is always established relative to some \emph{agent} \(a\) (e.g. a person): given some
proposition \(\phi\), \(K_a \phi\) reads that ``the agent \(a\) knows \(\phi\)''. The basic syntax
of first-order epistemic logic for propositional atoms \(p\) is given in \cref{fig:epistemic-logic}.
Disjunction, implication, bi-implication, and other familiar connectives may be defined from
negation and conjunction in the usual manner.

\begin{figure}[ht]
  \centering
  \[\begin{array}{rcl}
    \phi, \psi & \Coloneqq &
      p \mid
      \lnot \phi \mid
      \phi \land \phi \mid
      K_a \phi \mid
      \forall x. \phi \mid
      \exists x. \phi
  \end{array}\]
  \caption{Syntax of epistemic logic.}
  \label{fig:epistemic-logic}
\end{figure}

Flavors of \emph{dynamic} epistemic logic additionally includes \emph{actions} that cause the
knowledge of agents to change. Concretely, an agent may \emph{announce} \(\phi\) to another agent
\(a\), written \([!\phi]_{a}\). We write \([!\phi]_{a} \psi\) to denote that after \(\phi\) is
announced to agent \(a\), \(\psi\) is true. For example, if Cheryl knows that Alice is exactly two
years older than Bob but knows neither person's age, and Alice informs Cheryl that she is 25 years
old, then Cheryl knows that Bob is 23 years old.

We now have an expressive logical system that can intuitively describe a wide variety of privacy and
security policies. For example, consider a web server (\(s\)) that may send messages to some
recipient (\(r\)) but must not send some data (\(\alpha\)). We may formulate \(\lnot K_r \alpha\) to
state that \(r\) does not come to know \(\alpha\).

We can also describe more complex policies. Suppose that an online service (\(s\)) must give a user
(\(u\)) their data (\(\alpha\)) if a request for it is made (\(y\)). We can describe this policy by
the implication \(K_s y \to K_u \alpha\). For a different user (\(u'\)), it should be that \(\lnot
K_{u'} \alpha\); of course, the announcement of \(\alpha\) to \(u'\) would be in violation of the
policy, as \([!\alpha]_{u'} K_{u'} \alpha\) is valid. If the data consists of only \(\beta\) and
\(\gamma\), that is, \(\alpha \leftrightarrow \beta \land \gamma\), another user may be permitted to
know one piece but not the other. Indeed, \(\lnot K_{u'} \alpha\) implies \(\lnot (K_{u'} \beta
\land K_{u'} \gamma)\), or, equivalently, \(\lnot K_{u'} (\beta \land \gamma)\).

In addition, we may describe \emph{public} and \emph{group} announcements, in which knowledge may be
announced to \emph{all} agents \(I\) (written \([!\phi] \psi\)) or some restricted group of agents
\(G \subseteq I\) (written \([!\phi]_G \psi\)), respectively. Group announcement may be seen as a
generalization of both public and individual announcement. There are additional inference rules
governing the distribution of the knowledge modality over universal and existential quantification
and implication. For the sake of brevity, we omit the axioms of the system.

In the rest of the proposal, we describe how this dynamic epistemic logic is adapted for
fine-grained static information flow control via the \Hermit{} library.

\section{The Hermit in Action}

\cref{fig:example} demonstrates the \Hermit's design. In this example, the application registers a
user by storing (\lstinline{store}) their username and password in some database. However, there is
a critical policy security: the password \emph{must be hashed} (by calling the \lstinline{hash}
function) before it is given to the database. Unfortunately, since the developer has forgotten to
call \lstinline{hash} and \lstinline{pwd} is passed to \lstinline{store}, this policy is not
respected.

To enforce the desired security policy, the developer uses three library-provided attribute macros:
\lstinline{agent}, \lstinline{ensure}, and \lstinline{forgets}.

\begin{figure}
  \centering
  \begin{lstlisting}
use hermit::{agent, ensure, forgets};

fn store(username: String, pwd_hash: String) {
  ...
}

#[agent(secret)]
#[forgets(unhashed)]
fn hash(unhashed: String) -> String {
  ...
}

#[agent(secret)]
#[ensure(forall a, !a.K(pwd)]
pub fn register(username: String) {
  let pwd = ...
  store(username, pwd)
}
  \end{lstlisting}
  \caption{\Hermit{} in action.}
  \label{fig:example}
\end{figure}

\paragraph{Declaring agents}
Using \lstinline{agent}, the developer annotates the \lstinline{register} and \lstinline{hash}
functions with the agent \lstinline{secret}. This means that the agent \lstinline{secret}
\emph{knows} all the \emph{data} (including the those of the parameters) in these functions, and all
the variables in these functions (including the parameters) belong to the agent \lstinline{secret}.

Notice that \lstinline{store} is \emph{not} annotated by \lstinline{agent}. \Hermit{} automatically
assigns it to an anonymous default agent (we will call this agent \lstinline{db} for the purposes of
explanation), and that agent knows all of the data in the function.

Before continuing, it is important to distinguish between \emph{data} (the objects themselves) and
\emph{variables} (which refer to data). A piece of data may be known by any number of agents, but
each variable belongs to a single agent. This distinction is vital for the information flow
analysis, which we describe later.

\paragraph{Declaring policies}
The developer then uses the \lstinline{ensure} attribute to declare the security policies of these
functions; these are epistemic obligations. In this case, the policy is that the password does not
leave this critical region, i.e. no agent but \lstinline{secret} knows the data of \lstinline{pwd}.
This policy is formulated as a universal quantification over all (other) agents: for any (other)
agent \lstinline{a}, it is not (\lstinline{!}) the case that agent \lstinline{a} comes to know the
data of \lstinline{pwd} (\lstinline{a.K(pwd)}). Using the symbols of \cref{sec:background-del},
\(\forall a, \lnot K_a \text{\lstinline{pwd}}\). In other words, code attached
to the agent \lstinline{secret} is prohibited from causing the password to leak into the code
attached to any other agent.

In general, \lstinline{ensure} assertions may refer to any parameter or any variable bound in the body
of the function. \lstinline{ensure} declarations may also appear on \lstinline{let} declarations or
even data itself; one could instead write either of the following:

\begin{lstlisting}[firstnumber=16]
#[ensure(forall a, !a.K(pwd)]
let pwd = ...
\end{lstlisting}

\begin{lstlisting}[firstnumber=16]
let pwd = #[ensure(pwd => forall a, !a.K(pwd)]
          ...
\end{lstlisting}

\paragraph{Enforcing policies}
\Hermit{} statically enforces policies by (1) computing information flow and (2) checking that all
\lstinline{ensure} assertions hold.

In the first step, \Hermit{} computes the flow, i.e. the all the set of \emph{forward dependencies},
for each piece of data with a \lstinline{ensure} annotation. In \cref{fig:example}, since
\lstinline{pwd} is passed to \lstinline{store}, \lstinline{pwd_hash} is a forward dependency of the
password.

Recall that each variable \lstinline{x} belongs to some agent \lstinline{a}.
\Hermit{} views all movement and copying of data as the \emph{announcement} of that data to the
corresponding agent of the variable. For example, when the data of \lstinline{pwd} is moved into
\lstinline{pwd_hash}, \Hermit{} understands this operation as the announcement of the data to the
agent \lstinline{db}! Hopefully it is clear that this announcement is in direct violation of the
policy given by the \lstinline{ensure} policy once the announcement is made, the agent
\lstinline{db} \emph{knows} the data of \lstinline{pwd}, but the policy states that no other agents
are permitted to know the password. \Hermit{} is able to compute this violation and report an error.

Formally, let \(F(\mlstinline{pwd}) = \{\mlstinline{pwd_hash}\} \cup D\) be the forward dependencies
of \lstinline{pwd}, where \(D\) is the forward dependencies of \lstinline{pwd_hash}. Suppose that
the control flow of \lstinline{pwd_hash} is contained within \lstinline{store}, i.e. all members of
\(D'\) (and thus \(F(\mlstinline{pwd})\)) are of agent \lstinline{db}.

Then, \Hermit{} computes the validity of
\[%
  [\mlstinline{pwd}]_\mlstinline{db} \forall a \in A^{*}(F(\mlstinline{pwd})), \lnot K_a \mlstinline{pwd}
\]%
where \(A^{*}(D) = \{A(x) \mid x \in D\}\) and \(A(x)\) is the agent of the variable \(x\). Since
\(A^{*}(F(\mlstinline{pwd})) = A^{*}(\{\mlstinline{pwd_hash}\} \cup D) =
\{A(\mlstinline{pwd_hash})\} \cup A^{*}(D) = \{\mlstinline{db}\}\), we have, equivalently,
\[%
  [\mlstinline{pwd}]_\mlstinline{db} \forall a \in \{\mlstinline{db}\}, \lnot K_a \mlstinline{pwd}
    =
  [\mlstinline{pwd}]_\mlstinline{db} \bigwedge_{\{\mlstinline{db}\}} \lnot K_a \mlstinline{pwd}
\]%
The resulting statement is clearly invalid:
\[%
  [\mlstinline{pwd}]_\mlstinline{db} \lnot K_\mlstinline{db} \mlstinline{pwd}
\]%

\paragraph{Declassification}
Let us now suppose that the developer correctly hashes the password using \lstinline{hash} before
passing it to \lstinline{store}:

\begin{lstlisting}[firstnumber=16]
let pwd = ...
store(username, hash(pwd))
\end{lstlisting}

Now, the flow of the data of \lstinline{pwd} includes the parameter \lstinline{unhashed} and, by
extension, any forward dependencies of \lstinline{unhashed}. Presumably, this includes the output of
the function. Since moves and copies correspond to announcements, the data of \lstinline{pwd} is
announced to the agent of \lstinline{unhashed} and other forward dependencies. Since
\lstinline{unhashed} also belongs to the agent \lstinline{secret}, this is a no-op.

However, the output of \lstinline{hash(pwd)} is moved into the parameter \lstinline{pwd_hash} of
\lstinline{store}, i.e. \Hermit{} considers the password to have been announced to the agent
\lstinline{db}, again violating the policy! This is undesirable; we would like the \emph{hashed}
password to enter the database.

\Hermit{} provides the \lstinline{forgets} attribute to allow the developer to declare that the
output of \lstinline{hash} is \emph{independent} of the data of the parameter \lstinline{unhashed}.
Consequently, the output is no longer considered a forward dependency of the data of
\lstinline{unhashed}, and thus no longer a forward dependency of the data of \lstinline{pwd}. When
the output is moved to \lstinline{store}, \Hermit{} considers it---but not the password---to have
been announced to the agent \lstinline{db}. Now, the policy is satisfied. In effect, the output of
\lstinline{hash} is declassified. In another view, agents to whom the output is announced know the
output but ``forget'' that it has anything to do with the password in \lstinline{unhashed}.

Needless to say, this attribute should probably be used sparingly and carefully. Luckily, \Hermit's
design provides some safeguards against improper use. Specifically, where \lstinline{forgets} is
used, the agent must already know the data. For instance, let us consider again the case
\cref{fig:example}, in which the password is not hashed. The unhashed password cannot be known by
the agent \lstinline{db} just by annotating \lstinline{store} with \lstinline{#[forgets(pwd_hash)]};
\Hermit{} will still detect a violation of the policy at the call to \lstinline{store} on line 17.
Of course, if \lstinline{store} is annotated with \lstinline{#[agent(secret)]}, all guarantees are
abandoned (in much the same way that declaring items \lstinline{unsafe} forsakes memory safety
guarantees).

The name \lstinline{forgets} is hopefully alarming enough to generate some caution. Further linting
processes could ensure that only trusted developers may apply the \lstinline{forgets} annotation. As
with the \lstinline{ensure} annotation, the \lstinline{forgets} annotation can also appear on
individual bindings or data.

\section{Implementation}

\Hermit{} is implemented as a library of procedural macros for Rust. Internally, information flow is
computed using the Flowistry library \cite{crichton_2022a}, and assertions are checked by
dispatching SMCDEL \cite{gattinger_2018a}, a symbolic model checker for dynamic epistemic logic.

\section{Evaluation}

In such a system, we are firstly concerned with correctness, which depends primarily on the
correctness of the Flowistry library \cite{crichton_2022a} and SMCDEL \cite{gattinger_2018a}.
However, given resource constraints, we focus in this work on the usability of \Hermit{} itself by
pursuing case studies of its application to existing systems.

We perform preliminary evaluation of \Hermit{}'s expressiveness and ease of use on WebSubmit
\cite{schwarzkopf_2022a}, an existing web application for collecting student homework submissions.
WebSubmit consists of approximately 1,000 lines of Rust code and uses the \cite{benitez_2024a} web
framework. In this small case study, we attempt to address the following questions:

\begin{itemize}
  \item How flexible is \Hermit{} in expressing realistic information flow control policies? Are
    there policies that are difficult or cumbersome to express in dynamic epistemic logic? Are there
    policies for which dynamic epistemic logic is well-suited?

  \item How much developer effort is required to apply \Hermit{}? We explore several different
    metrics, such as annotations-per-function or annotations-per-module.
\end{itemize}

\bibliographystyle{plain}
\bibliography{main.bib}

\end{document}
